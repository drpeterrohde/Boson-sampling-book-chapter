\documentclass[aps,pra,twocolumn,amsmath,amssymb,nofootinbib,superscriptaddress]{revtex4}

\newcommand{\bra}[1]{\langle#1|}
\newcommand{\ket}[1]{|#1\rangle}
\newcommand{\op}[2]{\hat{\textbf{#1}}_{#2}}
\newcommand{\dagop}[2]{\hat{\textbf{#1}}_{#2}^\dag}
\newcommand{\keith}[1]{{\color{cyan}{#1}}}
\newcommand{\peter}[1]{{\color{blue}{#1}}}
\newcommand{\remove}[1]{{\color{red}{#1}}}

\usepackage[pdftex]{graphicx}
\usepackage{mathrsfs}
\usepackage[colorlinks]{hyperref}

\begin{document}

\bibliographystyle{apsrev}

%
% Title
% 

\title{An introduction to boson-sampling}

%
% Authors
%

\author{In no particular order}

\author{Peter P. Rohde}
\email[]{dr.rohde@gmail.com}
\homepage{http://www.peterrohde.org}
\affiliation{Centre for Engineered Quantum Systems, Department of Physics and Astronomy, Macquarie University, Sydney NSW 2113, Australia}

\author{Bryan Gard}

\author{Keith R. Motes}

\author{Jonathan P. Dowling}

\date{\today}

\frenchspacing

%
% Abstract
%

\begin{abstract}
\end{abstract}

\maketitle

\section{Introduction (Bryan)}

\subsection{Motivation for LOQC and boson-sampling (Bryan)}

\section{Intro to LOQC (Bryan)}

\section{Why is LOQC hard? (Bryan)}

\section{The boson-sampling formalism}

Unlike full LOQC, which requires active elements, the boson-sampling model is strictly passive, requiring only single-photon sources, passive linear optics (i.e beamsplitters and phase-shifters), and photodetection. No quantum memory or feedforward is required.

We begin by preparing an input state comprising $n$ single photons in $m$ modes,
\begin{eqnarray} \label{eq:input_state}
\ket{\psi_\mathrm{in}} &=& \ket{1_1,\dots,1_n,0_{n+1},\dots,0_m} \nonumber \\
&=& \hat{a}^\dag_1 \dots \hat{a}^\dag_n \ket{0_1,\dots,0_m},
\end{eqnarray}
where $\hat{a}^\dag_i$ is the photon creation operator in the $i$th mode. It is assumed that the number of modes scales quadratically with the number of photons, \mbox{$m=O(n^2)$}. The input state is evolved via a passive linear optics network, which implements a unitary map on the creation operators,
\begin{equation} \label{eq:Utransform}
\hat{U}\hat{a}_i^\dag\hat{U}^\dag = \sum_{j=1}^m U_{i,j} \hat{a}_j^\dag,
\end{equation} 
where $U$ is a unitary matrix characterizing the linear optics network. It was shown by Reck \emph{et al.} \cite{bib:Reck94} that any $U$ may be efficiently decomposed into $O(m^2)$ optical elements. The output state is a superposition of the different configurations of how the $n$ photons could have arrived in the output modes,
\begin{equation}
\ket{\psi_\mathrm{out}} = \sum_S \gamma_S \ket{n_1^{(S)},\dots,n_m^{(S)}},
\end{equation}
where $S$ is a configuration, $n_i^{(S)}$ is the number of photons in the $i$th mode associated with configuration $S$, and $\gamma_S$ is the amplitude associated with configuration $S$. The probability of measuring configuration $S$ is given by \mbox{$P_S = |\gamma_S|^2$}. The full model is illustrated in Fig.~\ref{fig:model}

\begin{figure}[!htb]
\includegraphics[width=0.7\columnwidth]{model}
\caption{The boson-sampling model. $n$ single photons are prepared in $m$ optical modes. These are evolved via a passive linear optics network $\hat{U}$. Finally the output statistics are sampled via coincidence photodetection. The experiment is repeated many times, reconstructing the output distribution $P_S$.} \label{fig:model}
\end{figure}

It was shown by Scheel \cite{bib:Scheel04perm} that the amplitudes $\gamma_S$ are related to matrix permanents,
\begin{equation}
\gamma_S = \frac{\mathrm{Per}(U_S)}{\sqrt{n_1^{(S)}!\dots n_m^{(S)}!}},
\end{equation}
where $U_S$ is an \mbox{$n\times n$} sub-matrix of $U$, and \mbox{$\mathrm{Per}(U_S)$} is the permanent of $U_S$.

Let us examine this relationship with the permanent more closely. Consider Fig.~\ref{fig:two_photon_perm}. Here the first two modes have single photons, with the remaining modes in the vacuum state. Let us consider the amplitude of measuring one photon at output mode 2 and another at output mode 3. Then there are two ways in which this could occur. Either the first photon reaches mode 2 and the second mode 3, or vice versa, i.e the photons pass straight through, or swap. Therefore there are \mbox{$2!=2$} ways in which the photons could reach the outputs. Thus, this amplitude may be written as,
\begin{eqnarray} \label{eq:coinProbEx}
\gamma_{\{2,3\}} &=& \underbrace{U_{1,2}U_{2,3}}_{\mathrm{walkers\ don't\ swap}} + \underbrace{U_{1,3}U_{2,2}}_{\mathrm{walkers\ swap}} \nonumber \\
&=& \mathrm{Per} \left[ {\begin{array}{cc}
   U_{1,2} & U_{2,2} \\
   U_{1,3} & U_{2,3} \\
  \end{array} } \right],
\end{eqnarray}
which is a \mbox{$2\times 2$} matrix permanent.

\begin{figure}[!htb]
\includegraphics[width=0.7\columnwidth]{two_photon_combinatorics}
\caption{Two-photon boson-sampling, where we wish to calculate the amplitude of measuring a photon at each of the output modes 2 and 3. There are two ways in which this may occur -- either the photons pass straight through, or swap, yielding a sum of two paths.} \label{fig:two_photon_perm}
\end{figure}

As a slightly more complex example, consider the three photon case shown in Fig.~\ref{fig:three_photon_perm}. Now we see that there are \mbox{$3!=6$} ways in which the three photons could reach the outputs, and the associated amplitude is given by a \mbox{$3\times 3$} matrix permanent,
\begin{eqnarray} \label{eq:coinProbEx3}
\gamma_{\{1,2,3\}} &=& U_{1,1}U_{2,2}U_{3,3} + U_{1,1}U_{3,2}U_{2,3} \nonumber \\
&+& U_{2,1}U_{1,2}U_{3,3} + U_{2,1}U_{3,2}U_{1,3} \nonumber \\
&+& U_{3,1}U_{1,2}U_{2,3} + U_{3,1}U_{2,2}U_{1,3}
\nonumber \\
&=& \mathrm{Per} \left[ {\begin{array}{ccc}
   U_{1,1} & U_{2,1} & U_{3,1} \\
   U_{1,2} & U_{2,2} & U_{3,2} \\
   U_{1,3} & U_{2,3} & U_{3,3} \\
  \end{array} } \right].
\end{eqnarray}

\begin{figure}[!htb]
\includegraphics[width=\columnwidth]{three_photon_combinatorics}
\caption{Thee photon boson-sampling, where we wish to calculate the amplitude of measuring a photon at each of the output modes 1, 2 and 3. There are now \mbox{$3!=6$} possible routes for this to occur.} \label{fig:three_photon_perm}
\end{figure}

In general, with $n$ photons, there will be $n!$ ways in which the photons could reach the outputs (assuming they all arrive at distinct outputs), and the associated amplitude will relate to an \mbox{$n\times n$} matrix permanent. Calculating matrix permanents is known to be \textbf{\#P}-complete, even harder than \textbf{NP}-complete, and the best known algorithm for calculating matrix permanents is by Ryser \cite{bib:Ryser63}, requiring \mbox{$O(2^n n^2)$} runtime. Thus, we can immediately see that if boson-sampling were to be classically simulated by calculating the matrix permanents, it would require exponential classical resources.

Because the number of modes scales quadratically with the number of photons, for large systems we are statistically guaranteed that all photons will arrive at different output modes. This implies that in this regime on/off (or `bucket') detectors will suffice, and photon-number resolution is not necessary, a further experimental simplification compared to full-fledged LOQC.

The number of configurations in the output modes scales as,
\begin{equation}
|S| = \binom{n+m-1}{n},
\end{equation}
which is exponential is $n$. Thus, with an `efficient' (i.e sub-exponential) number of trials, we are unlikely to sample from a given configuration more than once. This implies that we are unable to determine any given $P_S$ with more than binary accuracy. Thus, boson-sampling does \emph{not} let us \emph{calculate} matrix permanents, as doing so would require determining amplitudes with a high level of precision, which would require an exponential number of measurements.

The experiment is repeated may times, each time performing a coincidence photodetection at the output modes. Thus, after each run we sample from the distribution $P_S$. This yields a so-called \emph{sampling problem}, whereby the goal is to sample a statistical distribution using a finite number of measurements. This is in contrast to well-known \emph{decision problems}, such as Shor's algorithm \cite{bib:Shor97}, which provide a well-defined answer to a well-posed question.

This sampling problem was shown by Aaronson \& Arkhipov to be a computationally hard problem. That is, reconstructing the statistical distribution at the output to the boson-sampling device is computationally hard. However, whilst shown to be computationally hard, no known applications for boson-sampling have been described. Thus, boson-sampling acts as an interesting proof-of-principle demonstration that linear optics can outperform classical computers, but, based on present understanding, does not solve a problem of practical interest.

\subsection{Sampling problems vs. decision problems (Peter)}

\subsection{Why is boson-sampling so much easier than linear optics quantum computing? (Peter)}

\subsection{Errors in boson-sampling (Johnny)}
Discuss the 1/poly(n) bound

\section{Boson-sampling and the Extended Church-Turing thesis}

Any model for quantum computation is subject to errors of some form. In the conventional circuit model, this includes errors such as dephasing. In linear optics, this includes photon loss and mode-mismatch. Let us consider a very generic error model for boson-sampling, where the single photon states are the desired single photon with probability $p$, otherwise are in some erroneous state \cite{bib:BSECT}. This erroneous state could, for example, comprise terms with the wrong photon number (such as loss or second order excitations), or mode-mismatch. Then our input state is of the form,
\begin{equation} \label{eq:error_model}
\hat\rho_\mathrm{in} =\left(\bigotimes_{i=1}^n[p\ket{1}\bra{1} + (1-p)\hat\rho_\mathrm{error}^{(i)}]\right) \otimes [\ket{0}\bra{0}]^{\otimes^{m-n}},
\end{equation}
where \mbox{$\hat\rho_\mathrm{error}^{(i)}$} may be different for each input mode $i$. This is an independent error model, whereby each state is independently subject to an error channel. $p$ stipulates the fidelity of the single photon states. When \mbox{$p=1$}, the states are perfect single photons, and when \mbox{$p<1$}, the state contain erroneous terms. We desire to sample from the distribution of Eq.~\ref{eq:input_state}, whereby none of the input states are erroneous. This occurs with probability $p^n$.

Let $P$ be the probability that upon performing boson-sampling we have sampled from the correct distribution, otherwise we sample from noise. The complexity proof provided by Aaronson \& Arkhipov only considered the regime where \mbox{$P>1/\mathrm{poly}(n)$}. Thus, for computational hardness, we require \mbox{$p^n > 1/\mathrm{poly}(n)$}. Clearly in the asymptotic limit of large $n$, this bound can never be satisfied for any $p<1$. Thus, with this independent error model, boson-sampling will always fail in the asymptotic limit.

Numerous authors \cite{bib:Broome20122012, bib:ShenDuan13, bib:AA13response, bib:Shchesnovich13, bib:Molmer13} have claimed that large-scale demonstrations of boson-sampling could provide elucidation on the validity of the Extended Church-Turing (ECT) thesis -- the statement that any physical system may be efficiently simulated on a Turing machine. However, it must be noted that the ECT thesis is by definition an asymptotic statement about arbitrarily large systems. Because the required error bound for boson-sampling is never satisfied in this limit, it is clear that boson-sampling cannot elucidate the validity of the ECT thesis as asymptotically large boson-sampling devices must fail under an independent error model.

This concern might be overcome in the future with either (1) a loosing of the error bound to $1/\mathrm{exp}(n)$, or (2) the development of fault-tolerance techniques for boson-sampling. However, to-date no such developments have been made. Thus, based on \emph{present} understanding, boson-sampling will not answer the question as to whether the ECT thesis is correct or not. However, this is distinct from the question `will boson-sampling yield \emph{post-classical} computation?'. The answer to this question may very well be affirmative, as this only requires a finite sized device, just big enough to beat the best classical computers.

\section{Boson-sampling with other classes of quantum optical states (Johnny)}

\section{How to build a boson-sampling device}

In this section we explain the basic components required to build a boson-sampling device. This device consists of three basic components: (1) single-photon sources; (2) linear optical networks; and, (3) photodetectors. Each of these present their own engineering challenges. There are a range of technologies that could be employed for each of these components. However, although boson-sampling is much easier to implement than full-scale LOQC, it remains challenging to build a post-classical boson-sampling device. 

\subsection{Photon sources}

\textbf{add in citations}

The first engineering challenge is to prepare an input state of the form of Eq.~\ref{eq:input_state}. This state may be generated using various photon source technologies. For a review of many of the photon sources see Ref.~\cite{bib:SourceAndDetectorReview}. Presently, the most commonly employed photon source technology is spontaneous parametric down conversion (SPDC).

The SPDC source works by first pumping a non-linear crystal with a laser source. With some probability one of the laser photons interacts with the crystal and emits an entangled superposition of photons across two output modes, the \emph{signal} and \emph{idler}. The output of an SPDC source is of the form,
\begin{equation} \label{SPDC}
\ket{\Psi_\mathrm{SPDC}} = \sqrt{1-\chi^2}\sum_{n=0}^{\infty}\chi^n\ket{n}_s\ket{n}_i,
\end{equation}
where $\chi$ is the squeezing parameter, $n$ is the number of photons, $s$ represents the signal mode, and $i$ represents the idler mode. For boson-sampling, we are interested in the $\ket{1}_s\ket{1}_i$ term of this superposition. The signal photons are measured by a photo-detector and because of the correlation in photon-number, we know that a photon is also present in the idler mode. The idler photons are then routed into one of the input ports of the boson-sampling device using a multiplexer \cite{bib:migdall2002tailoring, bib:LPOR201400027, bib:ma2011experimental}.

There are several problems associated with SPDC sources, which limit the scalability of boson-sampling. The major problem is higher order photon-number terms. In the boson-sampling model we only want the $\ket{1}_s\ket{1}_i$ term, which is far from deterministic. The SPDC source is going to emit the zero-photon term with highest probability, with exponentially decreasing higher order terms. If the heralding photodetector does not have unit efficiency, then the heralded mode may contains higher order photon-number terms.

It was recently shown my Motes \emph{et al.} \cite{bib:motes2013spontaneous} that SPDC sources are scalable in the asymptotic limit for boson-sampling. Specifically, if the photodetection efficiency is sufficient to guarantee post-selection at the output of the boson-sampling device with high probability then the heralded SPDC photons also have asymptotically high fidelity. The boson-sampling architecture with multiplexing is shown in Fig.~\ref{fig:multiplexing}. 

\begin{figure}[!htb]
\includegraphics[width=0.7\columnwidth]{multiplexing}
\caption{Boson-sampling architecture using SPDC sources with an active multiplexer. $N$ sources operate in parallel, each heralded by an inefficient single-photon number-resolving detector. It is assumed that \mbox{$N\gg n$}, which guarantees that at least $n$ photons will be heralded. The multiplexer dynamically routes the successfully heralded modes to the first $n$ modes of the unitary network $U$. Finally, photodetection is performed and the output is post-selected on the detection on all $n$ photons.}
\label{fig:multiplexing}
\end{figure}

Another problem is that photons from SPDC sources have uncertainty in their temporal distribution. If a boson-sampling device is built using multiple SPDC sources it is difficult to temporally align each of the $n$ photons entering the device. The error term associated with this scales exponentially with $n$, yielding an error model consistent with Eq.~\ref{eq:error_model}, which undermines operation in the asymptotic limit. 

\subsection{Linear optical networks}

After the input state has been prepared it is evolved via a linear optics network, $\hat{U}$. $\hat{U}$ transforms the input state as per Eq.~\ref{eq:Utransform} and may be completely characterized before the experiment using coherent state inputs \cite{bib:PhysRevLett.73.58}. $\hat{U}$ is composed of an array of discrete elements, namely, beam-splitters and phase shifters. A beamsplitter with phase-shifters may be represented as a two-mode unitary of the form \cite{bib:GerryKnight05},
\begin{equation} \label{eq:BS}
U_{\mathrm{BS}}(t) = \left( \begin{array}{cc}
e^{i(\alpha-\frac{\beta}{2}-\frac{\gamma}{2})}\mathrm{cos}\left(\frac{\delta}{2}\right) & -e^{i(\alpha-\frac{\beta}{2}+\frac{\gamma}{2})}\mathrm{sin}\left(\frac{\delta}{2}\right)  \\
e^{i(\alpha+\frac{\beta}{2}-\frac{\gamma}{2})}\mathrm{sin}\left(\frac{\delta}{2}\right) & e^{i(\alpha+\frac{\beta}{2}+\frac{\gamma}{2})}\mathrm{cos}\left(\frac{\delta}{2}\right)
\end{array} \right), 
\end{equation}
where \mbox{$0\leq\alpha\leq2\pi$} and \mbox{$0\leq\{\beta,\gamma,\delta\}\leq\pi$} are arbitrary phases. It was shown by Reck \emph{et al.} \cite{bib:Reck94} that an arbitrary unitary $\hat{U}$ can be constructed with $O(m^2)$ optical elements, where $m$ is the number of inputs to the boson-sampling device.

For a $\hat{U}$ that implements a classically hard problem one would need hundreds of discrete optical elements. Constructing an arbitrary $\hat{U}$ using the traditional linear optical approach of setting and aligning each optical element would be extremely cumbersome. Thus, using discrete optical elements is not a very promising route towards scalable boson-sampling.

One method to simplify the construction of the linear optics network is to use integrated waveguides. Quantum interference was first demonstrated with this technology by Peruzzo \emph{et al.} \cite{bib:peruzzo2011multimode}. This technology requires more frugal space requirements, is more optically stable, and far easier to manufacture, allowing the entire linear optics network to be integrated onto a small chip \cite{bib:Politi02052008, bib:matthews2009, bib:Politi04092009}. The main issue with integrated waveguides is achieving sufficiently low loss rates inside of the waveguide and in the coupling of the waveguide to the photon-sources and photo-detectors. Presently, the loss rates in these devices are extremely high and post-selection upon $n$ photons at the output occurs with very low probability. It is foreseeable that photon-sources and photodetectors will eventually be integrated into the waveguide which would eliminate coupling loss rates, substantially improving scalability.   

Another potential route to simplifying the linear optical network is to use time-bin encoding in a loop based architecture \cite{bib:motes2014scalable}. The major advantage of this architecture is that it only requires two delay loops, two on/off switches, and one controllable phase-shifter as shown in Fig.~\ref{fig:fiber_loop}. This possibility eliminates the problem of aligning hundreds of optical elements and has fixed experimental complexity, irrespective of the size of the boson-sampling device. A major problem with this architecture however is that it remains difficult to control a dynamic phase-shifter with high fidelity at a rate that is on the order of the time-bin width $\tau$.

\begin{figure}[!htb]
\includegraphics[width=0.7\columnwidth]{fiber_loop}
\caption{Time-bin encoding architecture for implementing a boson-sampling device. Single photons arrive in a train of time-bins instead of in spatial modes. Each time-bin corresponds to spatial modes in the boson-sampling scheme and are separated by time $\tau$. The photon train is coupled into the loop by the first switch. The photons then transverse the inner loop such that each time-bin may interact. The first (last) photon is coupled completely in (out). The outer loop allows an arbitrary number of the smaller loops to be applied consecutively which is determined by the third switch. Finally, the photon train is measured at the output using time-resolved detection.}
\label{fig:fiber_loop}
\end{figure}

\subsection{Photodetection}

The final requirement in the boson-sampling device is sampling the output distribution. With linear optics this is done using photo-detectors. For a review on photodetection see Ref.~\cite{bib:SourceAndDetectorReview}.  

There are two general types of photo-detectors -- photon-number resolving detectors and bucket detectors. The former counts the number of incident photons. These are much more difficult to make and more expensive in general than bucket detectors. Bucket detectors, on the other hand, simply trigger if any non-zero number of photons are incident on the detector.

As discussed earlier, in the limit of large boson-sampling devices, we are statistically guaranteed that we never measure more than one photon per mode, since the number of modes scales as \mbox{$m=O(n^2)$}. Thus, bucket detectors are sufficient for large boson-sampling devices, a significant experimental simplification compared to universal LOQC protocols.

Some of the various types of photo-detectors are:

Number-resolving:

Bucket:


\begin{itemize}
\item List different types of detectors
\item don't need to be number resolving.
\item be sure to describe issues to overcome
\end{itemize}

\section{Conclusion (Jon)}

%
% Acknowledgments
%

\begin{acknowledgments}
This research was conducted by the Australian Research Council Centre of Excellence for Engineered Quantum Systems (Project number CE110001013).
\end{acknowledgments}

%
% Bibliography
%

\bibliography{bibliography}

\end{document}
