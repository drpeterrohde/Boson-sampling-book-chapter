\documentclass[aps,pra,twocolumn,amsmath,amssymb,nofootinbib,superscriptaddress]{revtex4}

\newcommand{\bra}[1]{\langle#1|}
\newcommand{\ket}[1]{|#1\rangle}
\newcommand{\op}[2]{\hat{\textbf{#1}}_{#2}}
\newcommand{\dagop}[2]{\hat{\textbf{#1}}_{#2}^\dag}
\newcommand{\keith}[1]{{\color{cyan}{#1}}}
\newcommand{\peter}[1]{{\color{blue}{#1}}}
\newcommand{\remove}[1]{{\color{red}{#1}}}

\usepackage[pdftex]{graphicx}
\usepackage{mathrsfs}
\usepackage[colorlinks]{hyperref}

\begin{document}

\bibliographystyle{apsrev}

%
% Title
%

\title{An introduction to boson-sampling}

%
% Authors
%

\author{In no particular order}

\author{Peter P. Rohde}
\email[]{dr.rohde@gmail.com}
\homepage{http://www.peterrohde.org}
\affiliation{Centre for Engineered Quantum Systems, Department of Physics and Astronomy, Macquarie University, Sydney NSW 2113, Australia}

\author{Bryan Gard}

\author{Keith R. Motes}

\author{Jonathan P. Dowling}

\date{\today}

\frenchspacing

%
% Abstract
%

\begin{abstract}
\end{abstract}

\maketitle

\section{Introduction}

\section{Motivation for linear optics quantum computing}

\section{Introduction to linear opticas quantum computing}

\section{Why is linear optics quantum computing hard?}

\section{Introduction to boson-sampling}

\subsection{The model}

\subsection{Sampling problems vs. decision problems}

\subsection{Why is boson-sampling so much easier than linear optics quantum computing?}

\section{Why is boson-sampling computationally hard?}

\subsection{The connection with matrix permanents}

\subsection{The complexity of matrix permanents}

\subsection{Errors in boson-sampling}
Discuss the 1/poly(n) bound

\section{Boson-sampling and the Extended Church-Turing thesis}
Why experimental boson-sampling will not elucidate the ECT thesis

\section{Boson-sampling with other classes of quantum optical states}

\section{How to build a boson-sampling device (Keith)}

In this section we explain what components are required to build a boson-sampling device. There are three basic components to building this device which are single-photon sources, linear optical networks, and photon detectors. In the lab these components have multiple options 

have their own issues to overcome, which are described below. 
There are various ways to piece together these components to make a boson-sampling device and in this section we describe many of the various possibilities. 
Although easier than LOQC it is still extremely challenging to build.


\subsection{Photon sources}

\begin{itemize}
\item Spontaneous Parametric Down Conversion (SPDC); List problems; insert equation; 
\item Quantum Dot
\end{itemize}

\subsection{Linear optics networks}

\begin{itemize}
\item Reck et al.
\item Waveguides; cite Alberto Peruzzo
\item Discrete elements; beam-splitters and phase shifters; give general equation of beam-splitter from Knight book.
\item in case of other types of bosons it need to be optical.
\item alignment.
\end{itemize}

\subsection{Photo-detection}

\begin{itemize}
\item List different types of detectors
\item don't need to be number resolving.
\end{itemize}

\section{Conclusion}

%
% Acknowledgments
%

\begin{acknowledgments}
This research was conducted by the Australian Research Council Centre of Excellence for Engineered Quantum Systems (Project number CE110001013).
\end{acknowledgments}

%
% Bibliography
%

\bibliography{paper}

\end{document}
