\documentclass[aps,pra,twocolumn,amsmath,amssymb,nofootinbib,superscriptaddress]{revtex4}

\newcommand{\bra}[1]{\langle#1|}
\newcommand{\ket}[1]{|#1\rangle}
\newcommand{\op}[2]{\hat{\textbf{#1}}_{#2}}
\newcommand{\dagop}[2]{\hat{\textbf{#1}}_{#2}^\dag}
\newcommand{\keith}[1]{{\color{cyan}{#1}}}
\newcommand{\peter}[1]{{\color{blue}{#1}}}
\newcommand{\remove}[1]{{\color{red}{#1}}}

\usepackage[pdftex]{graphicx}
\usepackage{mathrsfs}
\usepackage[colorlinks]{hyperref}

\begin{document}

\bibliographystyle{apsrev}

%
% Title
%

\title{An introduction to boson-sampling}

%
% Authors
%

\author{In no particular order}

\author{Peter P. Rohde}
\email[]{dr.rohde@gmail.com}
\homepage{http://www.peterrohde.org}
\affiliation{Centre for Engineered Quantum Systems, Department of Physics and Astronomy, Macquarie University, Sydney NSW 2113, Australia}

\author{Bryan Gard}

\author{Keith R. Motes}

\author{Jonathan P. Dowling}

\date{\today}

\frenchspacing

%
% Abstract
%

\begin{abstract}
\end{abstract}

\maketitle

\section{Introduction}

\section{Motivation for linear optics quantum computing}

\section{Introduction to linear opticas quantum computing}

\section{Why is linear optics quantum computing hard?}

\section{Introduction to boson-sampling}

\subsection{The model}

\subsection{Sampling problems vs. decision problems}

\subsection{Why is boson-sampling so much easier than linear optics quantum computing?}

\section{Why is boson-sampling computationally hard?}

\subsection{The connection with matrix permanents}

\subsection{The complexity of matrix permanents}

\subsection{Errors in boson-sampling}
Discuss the 1/poly(n) bound

\section{Boson-sampling and the Extended Church-Turing thesis}
Why experimental boson-sampling will not elucidate the ECT thesis

\section{Boson-sampling with other classes of quantum optical states}

\section{How to build a boson-sampling device (Keith)}

In this section we explain what components are required to build a linear optical boson-sampling device. There are three basic components to building this device which are single-photon sources, linear optical networks, and photon detectors. These components each have their own issues to overcome, which are described below. Also, there are multiple options for each of these and thus multiple ways of piecing a boson-sampling device together. We we describe many of these various possibilities in this section. Although boson-sampling is much easier to implement than full scale LOQC it remains extremely challenging to build a post-classical boson-sampling machine. 

\subsection{Photon sources}

Boson-sampling begins with by preparing the input state $\ket{\Psi_{\mathrm{in}}}=\ket{1,1,\dots,1,0,0,\dots,0}$. Using linear optics the input state is composed of $n$ single photons which can be generated using various types of available photon sources. The most promising candidate is spontaneous parametric down conversion (SPDC). 

The SPDC source works by pumping a non-linear crystal $(\chi^2)$ with a laser source. With some probability one of the laser photons interacts with the crystal and gets split into an entangled superposition of idler and signal photons. The signal photon is measured by a photo-detector. With a successful click we know that the idler photon exists, which then gets routed into one of the ports in the boson-sampling device. The output state of the SPDC source is,
\begin{equation} \label{SPDC}
\ket{\Psi_{SPDC}} = \sqrt{1-\chi^2}\sum_{n=0}^{\infty}\chi^n\ket{n}_s\ket{n}_i,
\end{equation}
where $\chi$ is the squeezing parameter, $n$ is the number of photons, $s$ represents the signal term, and $i$ represents the idler term.

There are several problems associated with SPDC sources. While SPDC sources are currently the most widely used sources in linear optical experiments they have problems which limit scaling boson-sampling to the post classical regime. The major problem is higher order photon terms. In the boson-sampling model we only want the $\ket{\Psi_{\mathrm{SPDC}}}\propto\chi\ket{1}_s\ket{1}_i$ term and getting a single photon is far from deterministic. In reality there is a large probability of getting the zero photon term and higher order terms with decreasing probability. Since there are $n$ required single photons this error term scales exponentially. 

Another problem is that SPDC sources is that there is uncertainty in the temporally distribution of the photons. If you build a boson-sampling device using multiple SPDC sources it is extremely difficult to temporally align each of the $n$ photons going into the device. This error term scales exponentially with the number of photons. Exponentially scaling errors totally renders your boson-sampling device useless since we can only tolerate polynomial error scaling with $n$. 

One way to get around some of these errors is to consider a multiplexing device \cite{}.

\begin{itemize}
\item Spontaneous Parametric Down Conversion (SPDC); List problems; insert equation; 
\item Quantum Dot
\item cite and describe our paper on SPDC scaling.
\end{itemize}

\subsection{Linear optics networks}

\begin{itemize}
\item Reck et al.
\item Waveguides; cite Alberto Peruzzo
\item Discrete elements; beam-splitters and phase shifters; give general equation of beam-splitter from Knight book.
\item in case of other types of bosons it need to be optical.
\item alignment.
\end{itemize}

\subsection{Photo-detection}

\begin{itemize}
\item List different types of detectors
\item don't need to be number resolving.
\end{itemize}

\section{Conclusion}

%
% Acknowledgments
%

\begin{acknowledgments}
This research was conducted by the Australian Research Council Centre of Excellence for Engineered Quantum Systems (Project number CE110001013).
\end{acknowledgments}

%
% Bibliography
%

\bibliography{paper}

\end{document}
